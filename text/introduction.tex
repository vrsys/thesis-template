% Showing citation commands
Let us get started by citing \cite{BeckKKF13}!
So what did \citeauthor{BeckKKF13} do in \citeyear{BeckKKF13}?
Good question!
% Showing hyperref reference commands
Maybe it is answered in \cref{introduction} on page~\pageref{introduction}?
\autoref{fig:a} is here to demonstrate how to include images.
\begin{figure}[bt]% bottom or top of page (for small figures/tables)
  \begin{center}{\huge\bfseries A}\end{center}
  \caption{The first letter in the Roman alphabet.}\label{fig:a}
\end{figure}
This is the well known Pythagorean theorem: $x^n + y^n = z^n$.
You can look it up on \url{https://www.wikipedia.org/}.
% \formatdate (or \formatdateshort)
This date does not exist: \formatdateshort{30}{2}{2014}
and is the same as \formatdate{30}{2}{2014}.
% An example table
And here is some table with some numbers (\autoref{tab:numbers})
which deserves to be on an extra page.

\begin{table}[p]% extra page (usually for large figures/tables)
  \caption{Tables have their captions above, figures below.}
  \begin{center}
    \begin{tabular}{lccc}\toprule
      \multicolumn{4}{c}{Some numbers}       \\\midrule
                        & 1999 & 2000 & 2001 \\\cmidrule(l){2-4}
      % cmidrule: A line from 2nd to 4th column, trimmed on the left hand side
      Distance (km)     & 23   & 18   & 42   \\
      Awesomeness (aws) & 3.2  & 8.1  & 2.4  \\\bottomrule
    \end{tabular}
  \end{center}\label{tab:numbers}%
\end{table}

Hypotheses and sub-hypotheses can be formulated like this:
\begin{hypothesis}
\label{hyp:test}
This is a hypothesis.
\end{hypothesis}
\begin{shypothesis}
\label{shyp:test}
This is a sub-hypothesis.
\end{shypothesis}
Hypothesis \cref{hyp:test} is very important and must be referenced in the text.

\section{Lorem} \label{sec:lorem}
\lipsum[1-3]

\subsection{Ipsum} \label{ssec:ipsum}
\lipsum[4-7]
